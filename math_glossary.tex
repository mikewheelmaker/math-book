\documentclass[a4paper]{book}

\usepackage{amsmath}
\usepackage{parskip}

\title{Mathematics Glossary}
\author{Radu-Mihai Rotariu}
\date{2022}

\begin{document}
\maketitle

\tableofcontents

\chapter{Introduction}
\section{Who is this book for?}
This book is for everyone. It contains (hopefully) all of mathematics, from the simplest of subjects
to the most advanced. It is not meant to be a textbook, but rather a training tool in the journey
of discovering and expanding one's mathematical knowledge and experience.

\section{What is this book about?}
This books goes through all of mathematics, from the most basic to the most advanced. It contains
the theory and an extensive amount of exercises, some solved and some unsolved. The solved exercises
are there to consolidate the theory and the unsolved ones are there to aid in the expansion of their
mathematical experience, with the difficulty ranging from easy to difficult.

\section{How to contribute?}
This book is meant to free mathematics from any costs associated with studying it. As such, this project
is completely free and available online both as the original LaTeX source and as a PDF file.

If you are a teacher or mathematician (amateur or professional) and you would like to contribute to
this project, you can do so in several ways:
\begin{itemize}
    \item You can send exercises via email. They will be added after a thorough review.
    \item You can fork this repository on GitHub and create a pull request with your changes.
    \item You can comment on the GitHub repository with your suggestions.
\end{itemize}

\chapter{A}
\section{Addition}
\subsection{Introduction}
\textbf{Addition}, usually signified by the \textbf{plus} symbol ($+$),
is one of the four basic operations of arithmetic.

The addition of two whole numbers results in the total amount or \textbf{sum} of those values.

For example, if we have three apples in a basket and we add two more apples, we now have five apples in total.
This can be represented using the mathematical expression $3 + 2 = 5$
(that is \emph{``three plus two equals five''}).

The main purpose of addition is to count. Addition allows the counting of items, people, real world objects,
or it can be defined and executed without referring to concrete objects, using abstractions called numbers.

Addition is a part of arithmetic, which is a branch of mathematics, but can also be found in algebra, 
also a branch of mathematics, where it can be used on vectors, matrices, and other algebraic structures.

Performing addition is one of the simplest numerical skills. Addition of very small whole numbers is
accessible to toddlers, with the most basic task, $1 + 1$, being performed by infants as young as 5 months.

Addition is taught in school in the primary grades using the decimal system, starting with single digits
and progressively moving to larger numbers.

\subsection{Notation and Terminology}
Addition is written using the plus sign ($+$) between the terms (in the infix notation). The result is expressed
with an equals sign ($=$). For example:

$1 + 1 = 2$ (\emph{``one plus one equals two}'')

$2 + 2 = 4$ (\emph{``two plus two equals four}'')

$1 + 2 = 3$ (\emph{``one plus two equals three}'')

$1 + 3 + 2 = 6$ (\emph{``one plus three plus two equals six}'')

$4 + 4 + 4 + 4 = 16$ (\emph{``four plus four plus four plus four equals sixteen}'')

There are also situations where addition is implied, even though no symbol appears. For example,
a whole number immediately followed by a fraction indicates the sum of the two, called a mixed number.
\[3\frac{1}{2} = 3 + \frac{1}{2} = 3.5\]
This notation can cause confusion sometimes, as it is not clear whether the juxtaposition denotes addition
or multiplication.

The sum of a series of related numbers can be expressed through the capital sigma notation ($\Sigma$),
which compactly denotes iteration:
\[\sum_{k=1}^{5}k^{2} = 1^{2} + 2^{2} + 3^{2} + 4^{2} + 5^{2} = 55\]

\[
    \left.
        \begin{array}{rcr}
        term & + & term \\
        summand & + & summand \\
        addend & + & addend \\
        augend & + & addend
        \end{array}
    \right\}
    = sum
\]

The numbers or the objects to be added in general addition are collectively reffered to as the \textbf{terms},
the \textbf{addends} or the \textbf{summands}; this terminollogy carries over to the summation of multiple terms.
This is to be distinguished from \textit{factors}, which are multiplied. Some authors call the first addend the
\textit{augend}. During the Renaissance, many authors didn't even consider the first addend an \textit{addend}
at all. Today, due to the commutative property of addition, \textit{augend} is rarely used and both terms are
generally called addends.

The above mentioned teminology is derived from Latin. ``Addition'' and ``add'' are English words derived from
the Latin verb \textit{addere}, which means ``to give to''. Using the geruntive suffix ``-nd'' result in
``addend'' (``thing to be added''). Likewise, adding the same suffix to the Latin verb \textit{augere}
(``to increase'') results in ``augend'' (``thing to be increased'').

``Sum'' and ``summand'' derive from the Latin noun \textit{summa} (``the highest'' or ``the top'') and the
related verb \textit{summare}. This is appropriate not only because the sum of two positive numbers is greater
than either, but because it was common for the ancient Greeks and Romans to add upward, so that the sum was
literally higher than the addends. Nowadays, the modern practice is to add downward.

The plus sign ``$+$'' is an abbreviation of the Latin word \textit{et} (``and''). It appears in mathematical
works dating back to at least 1489.

\subsection{Interpretations}

\subsection{Properties}

\subsection{Addition in different bases}

\subsection{Exercises}

\subsection{Related Topics}


\chapter{B}

\chapter{C}

\chapter{D}

\chapter{E}

\chapter{F}

\chapter{G}

\chapter{H}

\chapter{I}

\chapter{J}

\chapter{K}

\chapter{L}

\chapter{M}

\chapter{N}

\chapter{O}

\chapter{P}

\chapter{Q}

\chapter{R}

\chapter{S}

\chapter{T}

\chapter{U}

\chapter{V}

\chapter{W}

\chapter{Y}

\chapter{Z}

\end{document}