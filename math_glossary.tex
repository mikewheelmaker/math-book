\documentclass[a4paper]{book}

\usepackage{amsmath}
\usepackage{parskip}

\title{Mathematics Glossary}
\author{Radu-Mihai Rotariu}
\date{2022}

\begin{document}
\maketitle

\tableofcontents

\chapter{Introduction}
\section{Who is this book for?}
This book is for everyone. It contains (hopefully) all of mathematics, from the simplest of subjects
to the most advanced. It is not meant to be a textbook, but rather a training tool in the journey
of discovering and expanding one's mathematical knowledge and experience.

\section{What is this book about?}
This books goes through all of mathematics, from the most basic to the most advanced. It contains
the theory and an extensive amount of exercises, some solved and some unsolved. The solved exercises
are there to consolidate the theory and the unsolved ones are there to aid in the expansion of their
mathematical experience, with the difficulty ranging from easy to difficult.

\section{How to contribute?}
This book is meant to free mathematics from any costs associated with studying it. As such, this project
is completely free and available online both as the original LaTeX source and as a PDF file.

If you are a teacher or mathematician (amateur or professional) and you would like to contribute to
this project, you can do so in several ways:
\begin{itemize}
    \item You can send exercises via email. They will be added after a thorough review.
    \item You can fork this repository on GitHub and create a pull request with your changes.
    \item You can comment on the GitHub repository with your suggestions.
\end{itemize}

\chapter{A}

\chapter{B}

\chapter{C}

\chapter{D}

\chapter{E}

\chapter{F}

\chapter{G}

\chapter{H}

\chapter{I}

\chapter{J}

\chapter{K}

\chapter{L}

\chapter{M}

\chapter{N}

\chapter{O}

\chapter{P}

\chapter{Q}

\chapter{R}

\chapter{S}

\chapter{T}

\chapter{U}

\chapter{V}

\chapter{W}

\chapter{Y}

\chapter{Z}

\end{document}